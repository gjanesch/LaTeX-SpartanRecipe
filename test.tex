\documentclass{article}
\usepackage{spartanrecipe}

\begin{document}

\recipename{Saltine-Toffee Candy}


\begin{recipedescription}
A simple, cheap, easy-to-vary dessert that can be made with a slightly unusual mix of ingredients - in particular, saltines.
\end{recipedescription}


\begin{recipestats}
\cookingtime{20 minutes} \\
\servings{About 40 servings} \\
\oventime{400\deg{F}, 6 minutes} \\
\stovetime{Medium heat, 3 minutes} \\
\source{\url{http://www.budgetbytes.com/2011/06/saltine-toffee-candy/}}
\end{recipestats}


\begin{ingredients}{}
40-50 & saltines \\
1 cup & dark brown sugar, packed \\
1 cup & unsalted butter \\
12 oz & chocolate chips 
\end{ingredients}

\begin{equipment}
Baking sheet, 14'' by 10'' minimum \\
Aluminum foil \\
Spatula or butter knife \\
Medium saucepan
\end{equipment}

\begin{procedure}
\item Preheat the oven to 400\deg{F}.
\item Line a sufficiently large baking sheet with foil. Place a layer of saltines over the surface; the exact number of saltines will be dependent on the size of the sheet.
\item In a large pan over medium heat, melt the sugar and butter together.  Let the mixture come to a boil, then set a timer and let it boil for three minutes.
\item After the three minutes, pour the mixture over the layer of saltines.  Use a spatula or butter knife to spread out and level the toffee as necessary. Place the baking sheet into the oven for six minutes.
\item After the six minutes, remove the baking sheet and turn off the oven (but leave it closed).  The surface of the toffee should be bubbling when removed; let these bubbles subside before continuing.
\item Once the bubbling has stopped, sprinkle the chocolate chips fairly evenly over the surface of the toffee.  Place the tray back in to the (still off) oven and leave it there until the chocolate chips are glossy (about a minute).  Remove the tray from the oven and spread the melted chocolate with a spatula or butter knife.
\item Let the candy come to room temperature.  Once it has, it can be moved to the refrigerator.  Break into pieces to eat.
\end{procedure}


\begin{notes}
\item The normal recipe calls for bittersweet chocolate, but dark chocolate also works very well.
\item The toffee can be made separately and used for recipes or purposes other than this recipe.
\item Boiling the toffee over lower heat for longer will make for a stickier, much less crunchy toffee.  It makes it much harder to separate from the foil, though, so it's not advisable.
\item Graham crackers can be substituted in for the saltines.  If you do this, it is recommended that a pinch of kosher salt be added to the toffee while it is boiling.
\end{notes}


\end{document}